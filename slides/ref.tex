\begin{frame}
    \begin{itemize}[<+-| alert@+>] % 当然,除了alert,手动在里面插 \pause 也行
    	\item This is a slide template created by latex for XJTLUers.
        \item Overleaf  \\ \url{https://www.overleaf.com/latex/templates/xjtlu-beamer-template/sfrvnnpcsmgh}
        \item GitHub \\ \url{https://github.com/yaoshanliang/XJTLU-Beamer-Template}
    \end{itemize}
    
    \note {Write your notes.\\}
    \begin{note}
        {Write your notes here}
    \end{note}
\end{frame}


\subsection{Blocks}
\begin{frame}{Blocks}
    % Blocks styles
    \begin{block}{Block I}
        Text
    \end{block}

    \begin{alertblock}{Block II}
        Text
    \end{alertblock}

    \begin{exampleblock}{Block III}
        Text
    \end{exampleblock}  
    
    \successbox{Success box}
    \alertbox{Alert box}
    \simplebox{Simple box}
 
\end{frame}


%% ---------------------------------------------------------------------------
\subsection{Alorgithms}
\begin{frame}{Algorithms (pseudocode)}
    \begin{algorithm}[H]
        \SetAlgoLined
        \LinesNumbered
        \SetKwInOut{Input}{input}
        \SetKwInOut{Output}{output}
        \Input{x: float, y: float}
        \Output{r: float}
        \While{True}{
          r = x + y\;
          \eIf{r >= 30}{
           ``O valor de $r$ é maior ou iqual a 10.''\;
           break\;
           }{
           ``O valor de $r$ = '', r\;
          }
         } 
         \caption{Algorithm Example}
    \end{algorithm}
\end{frame}

\subsection{Equations}

\begin{frame}{Equation}
    \begin{block}{Equation without numbers} 
        \begin{equation*}
            J(\theta) = \mathbb{E}_{\pi_\theta}[G_t] = \sum_{s\in\mathcal{S}} d^\pi (s)V^\pi(s)=\sum_{s\in\mathcal{S}} d^\pi(s)\sum_{a\in\mathcal{A}}\pi_\theta(a|s)Q^\pi(s,a)
        \end{equation*}
    \end{block}
%    \begin{exampleblock}{Multiple equations\footnote{If containing text in equations,use $\backslash$mathrm\{\} or $\backslash$text\{\}}}
%       
%        \begin{align}
%            Q_\mathrm{target}&=r+\gamma Q^\pi(s^\prime, \pi_\theta(s^\prime)+\epsilon)\\
%            \epsilon&\sim\mathrm{clip}(\mathcal{N}(0, \sigma), -c, c)\nonumber
%        \end{align}
%    \end{exampleblock}
\end{frame}

\begin{frame}
    \begin{block}{Equation with numbers}
        % Taken from Mathmode.tex
        \begin{multline}
            A=\lim_{n\rightarrow\infty}\Delta x\left(a^{2}+\left(a^{2}+2a\Delta x+\left(\Delta x\right)^{2}\right)\right.\label{eq:reset}\\
            +\left(a^{2}+2\cdot2a\Delta x+2^{2}\left(\Delta x\right)^{2}\right)\\
            +\left(a^{2}+2\cdot3a\Delta x+3^{2}\left(\Delta x\right)^{2}\right)\\
            +\ldots\\
            \left.+\left(a^{2}+2\cdot(n-1)a\Delta x+(n-1)^{2}\left(\Delta x\right)^{2}\right)\right)\\
            =\frac{1}{3}\left(b^{3}-a^{3}\right)
        \end{multline}
    \end{block}
\end{frame}

%% ---------------------------------------------------------------------------
% This frame show an example to insert multi-columns
\subsection{Multi-columns}
\begin{frame}{Multi-columns}
    \begin{columns}{}
        \begin{column}{0.5\textwidth}
            \justify
            É possível colocar mais de uma coluna utilizando os comandos de $\backslash$begin\{column\}\{\} e $\backslash$end\{column\}
        \end{column}
        \begin{column}{0.5\textwidth}
            \justify
            Porém, o espaçamento deve ser proporcional entre as colunas para que estas colunas não entrem em coflito. O espaçamento é dado pelo segundo argumento do $\backslash$begin.
        \end{column}
    \end{columns}   
    \begin{columns}{}
        \begin{column}{0.5\textwidth}
            \justify
            É possível colocar mais de uma coluna utilizando os comandos de $\backslash$begin\{column\}\{\} e $\backslash$end\{column\}
        \end{column}
        \begin{column}{0.5\textwidth}
            \justify
            Porém, o espaçamento deve ser proporcional entre as colunas para que estas colunas não entrem em coflito. O espaçamento é dado pelo segundo argumento do $\backslash$begin.
        \end{column}
    \end{columns}     
\end{frame}

\begin{frame}[allowframebreaks]
    \frametitle{Reference}
    \printbibliography
\end{frame}
