% \newpage
% \section{Reflection on Learning} \label{sec:reflection}

% Throughout the course of the final year project, I have gained invaluable experience and insight into the challenges and rewards of research and development. The process has provided me with opportunities to refine my technical skills and enhance my understanding of DRL. However, there have also been areas where I could have improved my approach, which I would like to address in this reflection on learning.

% \subsection{Improvement in logging practices}

% One challenge I faced during the project was reproducing issues that had previously occurred. This was mainly due to insufficient logging and documentation of my work as I progressed. Although I used version control system to maintain the development of the software, I realised the importance of maintaining a detailed record of the steps taken, the hyperparameters used, and the outcomes observed, as it would have allowed me to efficiently identify and address problems, especially for repeating experiments.

% In future projects, I will log progresses in daily basis to ensure that any issues encountered can be easily reproduced and resolved. Figure \ref{fig:daliy-log} is a typical example.

% \begin{figure}[htbp]
%    \centering
%    \includegraphics[width=\textwidth]{a-rl-daliy-log.jpg}
%    \caption{A typical daily log \cite{ref:reproducing-drl}.}
%    \label{fig:daliy-log}
% \end{figure}

% \subsection{Comprehensive specification report}

% With hindsight, I should have invested more time and effort into producing a thorough and detailed specification report. This would have not only provided a more solid foundation for my project but also streamlined the process of preparing the final report. Much of the information and content from the specification report could have been directly incorporated into the final document, saving time and effort in the long run.

% For future endeavours, I will focus on creating a comprehensive specification report, as it will contribute significantly to the overall success of my projects.

% \subsection{Advance report drafting}

% I learned the importance of writing the report earlier. Due to time constraints and other obligations, I found myself rushing to complete the final report towards the end of the project. This made the process stressful and challenging and resulted in a greater likelihood of errors or omissions in the final document. I realised that starting early and allocating more time to report writing could have prevented this situation.

% Therefore, in future projects, I will set aside more time to work on reports to ensure that I am not rushing to complete them towards the end.

% \subsection{Foundational knowledge acquisition}

% During the course of my project, I devoted significant time to reading research papers in order to gain a deeper understanding of the subject matter. However, I discovered that this approach did not provide me with a solid foundation in the fundamental concepts and principles of the field. In retrospect, I should have allocated more time to studying textbooks and other resources that offered a comprehensive overview of the basics before diving into the specialised research papers.

% For future projects, I will adopt a more balanced approach by first focusing on mastering the foundational knowledge through textbooks, lectures, and online courses. This will ensure that I have a strong grasp of the essential concepts and can better contextualise the findings and insights presented in research papers.

%%%% Less Personal

\newpage
\section{Reflection on Learning} \label{sec:reflection}

Throughout the course of the final year project, valuable experience and insight were gained into the challenges of research and development. The process provided opportunities to refine technical skills and enhance understanding of DRL. However, there were also areas where improvements could have been made, which will be addressed in this reflection on learning.

\subsection{Improvement in logging practices}

One challenge encountered during the project was reproducing issues that had previously occurred. This was mainly due to insufficient logging and documentation of work as it progressed. Although a version control system was used to maintain the development of the software, the importance of maintaining a detailed record of the steps taken, the hyperparameters used, and the outcomes observed became apparent. Such documentation would have allowed for efficient identification and addressing of problems, especially when repeating experiments.

In future projects, logging progress on a daily basis will be implemented to ensure that any issues encountered can be easily reproduced and resolved. Figure \ref{fig:daliy-log} provides a typical example.

\begin{figure}[htbp]
\centering
\includegraphics[width=\textwidth]{a-rl-daliy-log.jpg}
\caption{A typical daily log \cite{ref:reproducing-drl}.}
\label{fig:daliy-log}
\end{figure}

\subsection{Comprehensive specification report}

In hindsight, more time and effort should have been invested into producing a thorough and detailed specification report. This would have not only provided a more solid foundation for the project but also streamlined the process of preparing the thesis. Much of the information and content from the specification report could have been directly incorporated into the thesis, saving time and effort in the long run.

For future endeavours, the focus at the early stage of the project will be on creating a comprehensive specification report, as it will contribute significantly to the final project report.

\subsection{Advance report drafting}

The importance of writing the report earlier was learned. Due to time constraints and other obligations, there was a rush to complete the thesis towards the end of the project. This made the process stressful and challenging and resulted in a greater likelihood of errors or omissions in the thesis. Starting early and allocating more time to report writing could have prevented this situation.

Therefore, in future projects, more time will be set aside to work on reports to ensure that they are not rushed towards the end.

\subsection{Foundational knowledge acquisition}

During the course of the project, significant time was devoted to reading research papers in order to gain a deeper understanding of the subject matter. However, this approach did not provide a solid foundation in the fundamental concepts and principles of the field. In retrospect, more time should have been allocated to studying textbooks and other resources that offered a comprehensive overview of the basics before diving into the specialised research papers.

For future projects, a more balanced approach will be adopted by first focusing on mastering the foundational knowledge through textbooks and online lecture series. This will ensure a strong grasp of the essential concepts and better contextualise the findings and insights presented in research papers.
