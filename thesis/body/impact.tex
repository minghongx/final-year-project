\section{Industrial Relevance, Real-world Applicability, and Scientific Impact} \label{sec:impact}

DRL has garnered immense attention from both the academic and industrial communities due to its remarkable performance in complex decision-making tasks. This breakthrough has led to significant advancements in several domains, including robotics, finance, healthcare, and natural language processing. In the field of robotics, DRL has emerged as an approach for autonomously acquiring complex behaviours from sensor data. One such application is mapless navigation, an essential capability for most mobile robots. This section discusses the industrial relevance, real-world applicability, and scientific impact of mapless navigation with DRL.

\subsection{Industrial relevance}

Traditionally, autonomous navigation relies on the simultaneous localisation and mapping (SLAM) algorithm combined with motion planning algorithms. This method requires a pre-built map of the navigation environment based on dense depth sensors. However, it has three main drawbacks: 1) the need for global knowledge, i.e., the map, to enable robust navigation; 2) the resource-intensive nature of maintaining the map; and 3) the precision of the map highly depends on the quality of sensor data \cite{ref:virtual2real-drl}. Rapid generation of navigation behaviours without global knowledge presents a challenge, and building a map based on sparse depth sensor data is nearly impossible \cite{ref:virtual2real-drl}.

A DRL-based mapless motion planner can overcome these difficulties with the help of lightweight localisation solutions, such as Wi-Fi localisation \cite{ref:virtual2real-drl}. Mapless navigation is particularly useful in scenarios where pre-built maps are not available, such as in unexplored environments, or in situations where the environment changes frequently. Unlike traditional methods of building and updating maps, the mapless motion planner requires only limited global knowledge, specifically the real-time target location with respect to the robot coordinate frame. This method greatly reduces hardware requirements but is mainly suitable for indoor environments.

\subsection{Real-world applicability}

DRL is centred around the idea of trial-and-error learning. However, this trial-and-error training process may lead to unexpected damage to the real robot for certain tasks, such as obstacle avoidance. Moreover, due to the sample inefficiency of current DRL algorithms, training in the real world takes time that exceeds acceptable levels. Training in virtual environments not only prevents damage to real objects but also accelerates the training process through simulation time acceleration and parallel training techniques.

Transferring learned policies from simulation to reality is not straightforward, as it often encounters the so-called sim-to-real gap. Various methodologies have been proposed to address this challenge, including: 1) zero-shot learning, which involves building a realistic simulator or obtaining enough simulated experience for direct real-world application; 2) domain randomisation, which involves randomising the simulation to cover the real distribution of the real-world data rather than carefully modelling real-world parameters; and 3) domain adaptation, which uses data from the source domain to improve a model on a different target domain \cite{ref:sim2real}. In robotic navigation, three promising methods have been proposed: curriculum learning \cite{ref:energy-efficient}, continual learning \cite{ref:sim2real}, and policy distillation for multiple tasks \cite{ref:sim2real}.

Although it is currently challenging to directly apply DRL-based mapless motion planners to real-world scenarios, it is not impossible.

\subsection{Scientific impact}

Mapless navigation has been widely used in research as a testbed for evaluating novel ideas. For instance, in \cite{ref:energy-efficient}, deep and spiking neural networks were evaluated for their energy efficiency in mapless navigation tasks using neuromorphic hardware. In another study, \cite{ref:huauv} utilised a Hybrid Unmanned Aerial Underwater Vehicle (HUAUV), capable of operating in both air and water environments, to replace the original differential wheeled robot.

Of particular interest is the deep reinforcement learning paradigm itself. Although DRL is not necessarily superior to traditional methods in all robotic tasks, many researchers are drawn to this paradigm due to its high generality and ability to solve high-dimensional problems. Research has already applied DRL to two types of high-dimensional robotic tasks, dexterous manipulation and legged robot locomotion, and achieved significant results.

\subsection{Conclusion}

In conclusion, mapless navigation with DRL has shown potential for overcoming the limitations of traditional SLAM with motion planning algorithms. Its industrial relevance lies in reducing hardware requirements and enabling rapid navigation behaviour generation without a precise map. However, the real-world applicability of DRL-based mapless motion planners is hindered by the sim-to-real gap, necessitating further research to bridge this gap and reduce the level of difficulty. Despite these challenges, DRL has been widely adopted in various research domains, and its generality and ability to solve high-dimensional problems make it an attractive approach for robotic tasks. As DRL continues to mature, it is expected to have a substantial impact on both scientific research and real-world applications, driving progress in autonomous navigation and other complex decision-making tasks.
