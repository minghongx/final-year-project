\section{Project Rationale and Industrial Relevance}

Traditional methods of motion planning build a map from sensor data and analyse this map to derive next actions. This type of methods is known as simultaneous localization and mapping (SLAM). Although this approach is reliable in large scale complex environments, the very high time and space resources required to maintain the maps make it too costly in small scale simple environments. DRL can perform motion planning in simple environment at low cost, though it does not guarantee reliability.

DRL-based motion planner is more competitive than SLAM for house robots that move only in tidy indoor environments. The computing systems of these robots usually have very little storage space and computing power with small battery capacity, which does not make them suitable for planning motions though mantaining an global obstacle map.

